\subsection{Cardiovascular system}
DPP4 has been commonly found to express in vascular endothelial cells and venous capillary beds.~\cite{Matheeussen2013,Shigeta2012} hDPP4 expression~\cite{Nemoto1999} and enzymatic activity has been identified in fibroblasts.~\cite{Ospelt2010}
\par 
Irregular incretin hormone activity can deregulate glucose-stimulated insulin secretion and can result direct or indirect complications on cardiovascular system.~\cite{Ussher2012} Due to the fact that DPP4 regulates a number of common incretin including GLP-1, inhibition DPP4 activity in type II diabetic therapy can result complex response in diabetic heart and coronary vasculature.~\cite{Ussher2014} Both DPP4\textsuperscript{-/-} and DPP4 knockout mice model studies have demonstrated a cardioprotective property in left anterior descending artery ligation.~\cite{Sauvé2010} A similar result has also been observed with wild type mice model with two dose sitagliptin over 24 hours period. The same protective effect upon DPP4 inhibition has also been observed in mice and rats models with cardiac ischemia.~\cite{Ussher2012} Those results proposed a cardio-protective profile when DPP4 enzymatic activity is suppressed. 
\par 

\subsubsection{Vascular System}
DPP4 inhibition in some clinical studies seems to lower blood pressure via vasodilation.~\cite{Kröller-Schön2012} This vasodialtory effects seems to be independent of circulating DPP4 substrates. Another clinical study using sitagliptin (at 50mg/d) and $\alpha$-glucosidase inhibitor voglibose (0.6mg/d) treating type II diabetes has found en