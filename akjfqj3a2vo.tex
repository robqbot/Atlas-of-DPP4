For~\textit{homo sapine} DPP4, prolyl oligopeptidase family region is located towards the C-terminus of the protein chain and spans from amino acid 561 to amino acid 763. Structurally, C-terminal domain 2$^{o}$ structures consist of 10 $\alpha$-helices, 8 $\beta$-strands. This region contains catalytic region (i.e. aa605-aa635)~\cite{Rawlings1991,Barrett1992,Polgár1992,Rawlings1994} with catalytic sequence "GW\textbf{S}YG"~\cite{Ogata_1992}, which coincides with serine proteases' consensus ("GX\textbf{S}XG"), therefore DPP4 is characterised as a serine protease. Further study on DPP4 structure reviewed the serine catalytic triad is made of Ser630, Asp708 and His740 with Ser630 at the active site. Structurally, one disulfide bond (Cys-649-Cys-762) stabilizes $\alpha$-helix structure (Met-746-Ser-764) and $\beta$-sheet at C-terminus. In DPP4 dimmer formation, $\alpha$-he
\par
This region is highly preserved through evolution and is suspected that the catalytic activity is essential to biological functions, therefore is selected under positive pressure. 