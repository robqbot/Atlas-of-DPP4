\subsection{Adipose tissue}
In human, DPP4 has been found in both sc and viseral adipose cells.~\cite{Lamers2011} In addition, DPP4 expression has also been observed on both dendritic cells and macrophages if they are in contact with visceral adipose deports.~\cite{Zhong2013} In disease models, elevated DPP4 expression has commonly been found, including obesity and inflammation~\cite{Zhong2013}. Increased DPP4 expression level is also commonly co-observed with increasing BMI, elevating baseline glycated hemoglobin (HbA\textsubscript{1c})~\cite{2011} and impaired glucoregulatory functions.~\cite{Zhong2013,Sell2013} Rising circulating DPP4 has also been noticed to correlate to severity of adipocyte hypertrophy, macrophage infiltrations and insulin resistance in type II diabetic induced obesity.~\cite{Sell2013}  
\par 
Animal study~\cite{Conarello2003} has revealed the resistance in high-fat dietary induced obesity in DPP4~\textsuperscript{-/-} mice model, which is believed due to increased net energy consumption evidenced by the elevated expression of uncoupling protein-1 and $\beta$-3 adrenergic receptor caused by the ablation of DPP4 activity. Later studies by~\citet{Bordicchia2012} and~\citet{Lockie2012} have further demonstrated DPP4 substrates BNP, GLP-1 and oxyntomodulin can promote respiration, energy expenditure and stimulate thermogenesis in interscapular brown adipose tissue (iBAT). In addition, further studies have demonstrated DPP4 has regulatory effects on peptides/proteins that exhibit can stimulate food-intake including NPY, PYY, enterostatin. However one study by~\citet{Drucker2006} failed to observe suppression effects on diatery intake and 