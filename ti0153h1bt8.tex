\subsubsection {Sialylation}
Sialic acid (Figure~\ref{664721}) are often used to refer a group of N- or O-substituted derivatives of neuraminic acid, which is a monosaccharide and has a nine-carbon backbone.~\cite{Vocadlo_2009} Sialylation is a chemical process, where a single monosaccharide forms covalent attachment with glycan component of glycosylated proteins. This sialic acid capping has been found to have profound molecular functions in glycoprotein function, stability, and metabolism; consequently correct sialylation is critical for cellular functions like signal recognition and cell adhesion.~\cite{Bhide_2016}
\par 
The degree of N-terminal sialylation~\cite{Stehling_1999} in DPP4 seems associating with numerous physiological conditions.  
silencing T-cell activity~\cite{K_hne_1996} 
\textit{In vitro} study conducted by~\citet{Schmauser1999} suggested that the degree of sialylation contributes to different isoelectric points (IP) for DPP4 and was responsible for extraordinary migration behavior under sulfate-polyacrylamide gel electrophoresis (SDS-PAGE).