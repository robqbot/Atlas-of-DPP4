\subsubsection {Sialylation}
Sialic acid (Figure~\ref{664721}) are often used to refer a group of N- or O-substituted derivatives of neuraminic acid, which is a monosaccharide and has a nine-carbon backbone.~\cite{Vocadlo_2009} Sialylation is a chemical process, where a single monosaccharide forms covalent attachment with glycan component of glycosylated proteins. This sialic acid capping has been found to have profound molecular functions in glycoprotein function, stability, and metabolism; consequently correct sialylation is critical for cellular functions like signal recognition and cell adhesion.~\cite{Bhide_2016}
\par 
N-terminal sialylation~\cite{Stehling_1999}
silencing T-cell activity~\cite{K_hne_1996} 
In vitro -desialylation of the enzyme and its resialylation confirmed that sialylation was responsible for this extraordinary migration behavior.