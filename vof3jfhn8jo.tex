\subsubsection{Antibody binding sites}
An array of antibodies have been demonstrated to be able to form crosslinks with DPP4 in this asparagine rich glycosylation subdomain, including but not limited to 1F7, BA5, TA1 and CB1. ~\cite{Hegen1997,Gaetaniello1998,De_Meester_1999} Uponforming DPP4-antibody interaction, TCR/CD3 mediated signal transduction has been trigged ~\cite{Hegen1997} through a series of tyrosine phosphorylation.
\par

\subsubsection{N-terminal region (aa108 - aa478)}

This domain defines DPP4 subfamily S9B, therefore is also referred as DPP IV N-terminal region. This domain collects more sequence variations through evolution in comparison with much more conserved C-terminal catalytic region. 
\par 
In addition to archetypal member DPP4, there are three additional enzymatically active members of DPP8, DPP9 and Fibroblast activation protein (FAP), as well as few do not possess hydrolytic property, which includes DPP6, DPP10. 


\subsubsection{ADA binding site}
Adenosine deaminase (ADA) is a deamination enzyme, which facilitates irreversible deamination of adenosine to inosine and 2' deoxyadenosine to 2' deoxyinosine~\cite{Franco_1998}. In the absent or weak ADA activity, the development of B-cell and T-cell are severely impaired, which in turn causing combined immunodeficiency diseases.~\cite{Cristalli2001} Interestingly, upon forming complex, both ADA and DPP4 possess individual enzymatic activity while co-stimulate T cell proliferation.~\cite{De_Meester_1999}  
\par 
ADA binding occurs at DPP4 cysteine-rich region,
Leu294 and Val341 on DPP4 have been demonstrated via mutagenesis to be critical for ADA-DPP4 binding.~\cite{Abbott_1999} Leu294 and Val341 are located on two short $\alpha$-helices (291-294, 341-343 respectively), those two $\alpha$-helices are suspected to play an important role in ADA binding. 
\par