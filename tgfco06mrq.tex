\subsection{Immune System}
DPP4/CD26 ubiquitously expressed in many cell types in both innate and adaptive immune system including T cells, activated B cells, activated natural killer cells and myeloid cells.~\cite{Abbott1994,Shingu2003,Hong1989,Gutschmidt1981,Dikov2004,Bühling1995,Tanaka1992,Gorrell1991} 

\subsubsection{NK cells}
Human DPP4 expression level in resting natural killer cells is inherently low, nevertheless, the DPP4 expression can be largely increased up to 30\% more upon stimulation from interleukin-2 (IL-2) and IL-15.~\cite{Bühling1994,Biuling1990,Yamabe1997} Evidence~\cite{Madueño1993,Roloff2005} suggested DPP4 involves in promoting cytokines production by NK cells. NK cell cytotoxicity study against tumour cells~\cite{Shingu2003,Frerker2009} also seems supporting DPP4's role in promoting cytokine secretion, where cytotoxicity has been prove diminished in DPP4\textsuperscript{-/-} rat model. Later two proteomic analysis studies~\cite{Topham2009,Casey2007} have found DPP4 presence on the membrane of secretory lysosomes in NK cells, which further supporting DPP4 has a promotional role in cytokine secretion. DPP4 seems to play a role in NK maturation, where DPP4\textsuperscript{-/-} rat and DPP4 knockout mice model both showed significantly higher percentage NK in leucocytes while total leucocytes counts decreased.~\cite{Yan2003,Frerker2009}

\subsubsection{Myeloid cells}
Evident~\cite{Zhong2013,Gliddon2002} found DPP4 expression on dendritic cells and monocytes/macrophages DPP4 expression has been demonstrated in rats~\cite{Epardaud2004, Ellingsen2007}, although some evident~\cite{Shah_2011} suggested DPP4 presence in human macrophages, the exclude evidence is still not available. In rats liver, DPP4 has been found in lysosomes from both stellate macrophages and microglia cells, the DPP4 expression can be elevated upon activation.~\cite{Fukui1990} 

\subsubsection{T cells}
DPP4(CD26) expression on resting T cells is scarce and tightly regulated. This lack of DPP4 expression property has been used as a selection markers in clinical practise to target human natural regulatory T cells.~\cite{Salgado2012,Garcia2014} Mostly observed DPP4 expression on resting T cell population is on CD4\textsuperscript{+} T memory helper cells~\cite{Gorrell1991, Morimoto1998} and CD8\textsuperscript{+} T\textsubscript{c} cells~\cite{Waumans2015,Hatano2013} and those subsets of T lymphocyte population have shown maximised response to antigen recalls~\cite{Waumans2015,Morimoto1998,Hatano2013}. More recent studies found human T helper type 17 (Th17) cells~\cite{Bengsch2012} and mucosal-associated invariant T cells (MAITs)~\cite{Sharma2015} both express high level of DPP4.

\subsubsection{B cells}
Similar to T cells, DPP4 (CD26) expression on human B cells is generally low in resting state and up to half human B cells population express DPP4 once activated.~\cite{B_hling_1995} Studies~\cite{B_hling_1995,Micouin1997} have found the suppression in DPP4 enzymatic activity by DPP4 inhibitors results reduction in B cell activation. However, whether DPP4 enzymatic activity plays a role in B cell development remain controversy. While one study~\cite{Yan2003} has found DPP4 enzymatic activity is important in facilitating immunoglobulin isotype switching in B cells in DPP4\textsuperscript{-/-} mice model, the other investigations denialed such correlation in both mice~\cite{Vora2009} and rat~\cite{Coburn1994} models. One study in F344-rat~\cite{Klemann2009} nevertheless, suggested a negative impact on B cell population\textit{in vivo} due to prolonged DPP4 inhibition. More recent study in human has also found the lake of DPP4 expression observed in B cell carcinoma