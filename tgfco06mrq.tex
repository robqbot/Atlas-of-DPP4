\subsection{Immune System}
DPP4/CD26 ubiquitously expressed in many cell types in both innate and adaptive immune system including T cells, activated B cells, activated natural killer cells and myeloid cells.~\cite{Abbott1994,Shingu2003,Hong1989,Gutschmidt1981,Dikov2004,Bühling1995,Tanaka1992,Gorrell1991} 

\subsubsection{NK cells}
Human DPP4 expression level in resting natural killer cells is inherently low, nevertheless, the DPP4 expression can be largely increased up to 30\% more upon stimulation from interleukin-2 (IL-2) and IL-15.~\cite{Bühling1994,Biuling1990,Yamabe1997} Evidence~\cite{Madueño1993,Roloff2005} suggested DPP4 involves in promoting cytokines production by NK cells. NK cell cytotoxicity study against tumour cells~\cite{Shingu2003,Frerker2009} also seems supporting DPP4's role in promoting cytokine secretion, where cytotoxicity has been prove diminished in DPP4\textsuperscript{-/-} rat model. Later two proteomic analysis studies~\cite{Topham2009,Casey2007} have found DPP4 presence on the membrane of secretory lysosomes in NK cells, which further supporting DPP4 has a promotional role in cytokine secretion. DPP4 seems to play a role in NK maturation, where DPP4\textsuperscript{-/-} rat and 

\subsubsection{Myeloid cells}
Evident~\cite{Zhong2013,Gliddon2002} found DPP4 expression on dendritic cells and monocytes/macrophages DPP4 expression has been demonstrated in rats~\cite{Epardaud2004, Ellingsen2007}, although some evident~\cite{Shah_2011} suggested DPP4 presence in human macrophages, the exclude evidence is still not available. In rats liver, DPP4 has been found in lysosomes from both stellate macrophages and microglia cells, the DPP4 expression can be elevated upon activation.~\cite{Fukui1990} 

\subsubsection{T cells}
\subsubsection{B cells}
