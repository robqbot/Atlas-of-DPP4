The human DPP4 IV gene span approximately around 70 kb and is found on the long arm of chromosome 2 (2q24.3). The gene consists of 26 exons with each size varies from 45 b to 1.4 kb.~\cite{Abbott1994} The 5'-flanking region has missing both TATA box and CAAT box, but a G-C rich region spans 300 bp which is suspected to promote binding with number of transcription factors like NF$\kappa$B, AP2 and Sp1~\cite{Abbott1994,Böhm1995} The transcription of DPP4 IV gene has been found in two mRNA forms with length of 2.8 and 4.2 kb respectively. RNA expression level seems to be regulated based on organ types~41-44
Several cytokines have been demonstrated to have regulatory effects on DPP4 expression including but not limited to interferon-$\gamma$(IFN-$\gamma$), tumour necrosis factor-$\alpha$ (TNF-$\alpha$), lipopolysaccharide 

\par
Human DPP4 protein consists of 766 amino acids~\cite{Bär2003,Misumi1992}, which has 6 structurally and functionally distinctive regions. DPP4 can form both homo-dimmer~\cite{Rasmussen2003,Oefner2003,Thoma2003} and hetero-dimmer with other S9b family members including fibroblast activation protein-$\alpha$ (FAP)~\cite{Ghersi2006}. Previous mutagenesis studies~\cite{Chien_2004,Chien_2006} have demonstrated the dimerisation is critical to DPP4 enzymatically activity. The same group has also demonstrated three regions in DPP4 monomer that contribute to the dimmer formation: transmembrane region~\cite{Chung_2010} and two domains within catalytic region, namely C-terminal loop~\cite{Chien_2004} and the extension part of the $\beta$-properller domain~\cite{Chien_2006}.  
\par 
Structurally, human DPP4 protein possesses three distinctive domains topology-wise, which includes a small cytoplasmic domain (aa1 - aa6), a type II transmembrane region (aa7 - aa29) as well as extracellular domain/soluble DPP4. 
\par