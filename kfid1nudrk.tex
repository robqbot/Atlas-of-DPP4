\subsubsection {Sialylation}
Sialyic acid are often used to refer a group of N- or O-substituted derivatives of neuraminic acid, which is a monosaccharide and has a nine-carbon backbone.~\cite{Vocadlo_2009} Sialylation is a chemical process, where a single monosaccharide forms covalent attachment with glycan component of glycosalated proteins. This sialyic acid capping has been found to have profond molecular functions in delivering correct cellular functions commonly in signal recognition and cell adhesion.~\cite{Bhide_2016} Sialylation plays an important role in delivering glycoprotein function, stability, and metabolism in particular are dependent upon correct sialylation.