\subsubsection {Sialylation}
Sialyic acid are often used to refer a group of N- or O-substituted derivatives of neuraminic acid, which is a monosaccharide and has a nine-carbon backbone.~\cite{Vocadlo_2009} Sialylation is a chemical process, where a single monosaccharide forms covalent attachment with glycan component of glycoproteins

Sialylation has profond cellular functions commonly in signal recognition and cell adhesion. Sialylation caps glycocans Glycoprotein function, stability, and metabolism in particular are dependent upon correct sialylation.