\subsection{DPP4 inhibitor selectivity}
While DPP4 possess rare post proline hydrolytic activity, DPP4 family enzyme FAP, DPP8 and DPP9 all share significant homology and have DPP4-like enzymatic potency.~\cite{Kirby_2010,Thornberry_2007} Although DPP8 and DPP9 share DPP4 enzymatic activity, study have suggested the potency is much lower. FAP also share portion of degradation repertoire with DPP4, members include GLP-1, GIP, Neuropeptide-Y (NPY), B-type natriuretic peptide (BNP), substance-P (SP) and peptide-YY (PYY).~\cite{Keane_2011} Despite lateral overlapping between DPP4 family members, DPP4 seems to be the only member that truncates chemokines.~\cite{Keane_2011}  
\par 
Study by~\citet{Lankas2005} employed threo- and allo-isoleucyl thiazolidide as DPP4 inhibitor has observed severe toxicities in dogs and rats. The allo-isoleucyl thiazolidide has shown 10 times more potency in inhibiting DPP8 and DPP9 than DPP4, which is suspected to be responsible for in this toxicity. This toxicity has been further demonstrated using a DPP8 and DPP9 selective inhibitor (2S,3R)-2-(2-amino-3-methyl-1-oxopentan-1-yl)-1,3-dihydro-2H-isoindole hydrochloride, in both rat and dog models. DPP8 and DPP9 selective inhibitors has also been demonstrated in the same study to reduce both cytokine release and T cell proliferation. However, study conducted by~\citet{Burkey2008} has not observe those adverse effects by using more potent DPP4 inhibitor (vildagliptin) in rodent models. Nevertheless, some DPP4 potent inhibitors including both vildagliptin and saxagliptin seem to cause skin necrosis and vasculitis in non-human primates but the exact mechanism remain mistri
\par 