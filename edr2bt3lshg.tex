\subsection{Transmembrane region (aa7- aa29)}

This region forms $\alpha$
DPP4 is anchored in the plasma membrane by this type II transmembrane protein spans 22 aa in length, which forms polytopic transmembrane $\alpha$-helical protein with hydrophobic core.~\cite{Hong_1990} Structurally, type II transmembrane protein is a bitopic protein and only span lipid bilayer single time with N-terminus on the cytoplasmic side and transmembrane helix sits next to the N-terminus and its C-terminal domain is typically targeted to the destination of ER lumen.~\cite{Luckey} It is highly likely that this transmembrane protein plays an important role in DPP4 maturation and transportation to the cellular surface. 
\par 
In addition, recent experiment by \citet{Chung_2010} using bioluminescence resonance energy transfer on transfected Chinese hamster ovary cells with mutant DPP4 revealed that this transmembrane region is significantly important for DPP4 dimmer formation, given the DPP4 only enzymatically active in its dimer form, this transmembrane region may potentially be used as a pharmaceutical target.
