\citet{Hopsu-Havu1966} has firstly discovered DPP4 in rat liver homogenates. The original interest was due to its capacity to hydrolise penultimate Proline, which is commonly found in collagens (-Gly-Pro-). Nevertheless, the later research has found that DPP4's inabilily to cleave Pro-Pro and Pro-Hyp dipeptides (commonly follows -Gly-Pro- in collagens), which baffled our understanding of DPP4's roles in collagen metabolism.
\par
Historically, DPP4 has been referred both as adenosine deaminase (ADA) binding protein~\cite{Kameoka_1993} and T cell activation antigen CD26~\cite{Fleischer_1994} in the context of immunology. DPP4 has been found to be widely expressed not only on the cell surface of immune cells including B cells, T cells, natural killer cells~\cite{Fleischer1987,Fleischer1988,Gorrell1991,Capuani2018,Bühling1994,Bühling1995} and lymphocytes~\cite{Gorvel1991}, but also on epithelial, endothelial of wide tissue types~\cite{Gorrell2001}. 
\par 
DPP4 has a variety of physiological roles in human biology, which are scattered across metabolism, immunity, endocrinology and cancer biology, among the all, DPP4 is arguably most widely known to be targeted by new generation inhibitory therapy in treating type II diabetics. This is due to DPP4's ability to hydrolysis and consequently inactivates incretins like glucagon-like peptide-1 (GLP-1), subnormal level of active incretins results malfunction glucose metabolism symptoms like type-II diabitics.  
\par 
