\citet{Hopsu-Havu1966} has firstly discovered DPP4 in rat liver homogenates. The original interest was due to its capacity to hydrolise penultimate Proline, which is commonly found in collagens (-Gly-Pro-). Nevertheless, the later research has found that DPP4's inabilily to cleave Pro-Pro and Pro-Hyp dipeptides (commonly follows -Gly-Pro- in collagens), which baffled our understanding of DPP4's roles in collagen metabolism.
\par
Historically, DPP4 has been referred both as adenosine deaminase (ADA) binding protein~\cite{Kameoka_1993} and T cell activation antigen CD26~\cite{Fleischer_1994} in the context of immunology. DPP4 has been found to be widely expressed on the cell surface of B cells, T cells and natural killer 