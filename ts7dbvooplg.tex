\subsection{Glycosylated (aa49 - aa324) and Cysteine-rich subdomain (aa325 - aa552)}

Structrually, this domain forms a unique 8 blades of $\beta$-propeller, which is suspected to form a hollow channel for substrate entry.

In terms of post-translational modifications, glycosylated subdomain is under heavy glycosylation, with up to 7 sites can be modified via N-Acetylglucosamine Asparagine (i.e. N-linked) at site 85, 92, 150, 219, 229, 281, 321.~\cite{Rasmussen2003,Thoma2003,Meng2010,Chen2009,Hiramatsu2003} N-glycosylation are believed important in protein folding and increasing structure stability.~\cite{Fan_1997} In addition, phosphorylation has been detected~\textit{in vivo} at site 44, 52, 256, 439, 440 in human sample.~\cite{Xia2008, Hornbeck2015, Mertins2014} There are also reported~\textit{in vivo} phosphorylation at site 211, 215, 219 in mouse leukemia T cell samples through mass spectrometry.~\cite{Hornbeck2015} There are studies suggested that the tyrosine phosphorylation is important for DPP4 binding with mannose 6-phosphate~\cite{Ikushima_2000} and interaction with c-Scr, HIV-Tat.~\cite{Bilodeau_2006,Fan_2012}
Acetylation has also been reported in this region at site 137 (133 for mouse), 376 (372 for mouse), 434. ~\cite{Lundby2012,Weinert2013} Although there are studies that suggested glycosylation of DPP4 (CD26) is important in DPP4 co-activated T cells proliferation~\cite{Ikushima_2000} and phosphorylation is important for DPP4 signal conducting process~\cite{Ishii_2001}, the details of the exact mechanisms are not yet to be reviewed.
\par

\subsubsection{Antibody binding sites}
An array of antibodies have been demonstrated to be able to form crosslinks with DPP4 in this asparagine rich glycosylation subdomain, including but not limited to 1F7, BA5, TA1 and CB1. ~\cite{Hegen1997,Gaetaniello1998,De_Meester_1999} Uponforming DPP4-antibody interaction, TCR/CD3 mediated signal transduction~\cite{Hegen1997 tyrosine phosphorylation 
\par

At Cystein-rich region, there are a total of 9 cysteine residues propagate through this region, which form four disulfide bonds (i.e. 328 $\Leftrightarrow$ 339, 385 $\Leftrightarrow$ 394, 44 $\Leftrightarrow$ 447, 454 $\Leftrightarrow$ 472)~\cite{Hiramatsu2003} that help bound $\beta$-strands into blade. 

\subsubsection{N-terminal region (aa108 - aa478)}

This domain defines DPP4 subfamily S9B, therefore is also referred as DPP IV N-terminal region.
\par 
In addition to archetypal member DPP4, there are three additional enzymatically active members of DPP8, DPP9 and Fibroblast activation protein (FAP), as well as few do not possess hydrolytic property, which includes DPP6, DPP10. 

\subsubsection{ADA binding site}
Adenosine deaminase (ADA) is a deamination enzyme, which facilitates irreversible deamination of adenosine to inosine and 2' deoxyadenosine to 2' deoxyinosine~\cite{Franco_1998}. In the absent or weak ADA activity, the development of B-cell and T-cell are severely impaired, which in turn causing combined immunodeficiency diseases.~\cite{Cristalli2001} Interestingly, upon forming complex, both ADA and DPP4 possess individual enzymatic activity while co-stimulate T cell proliferation.~\cite{De_Meester_1999}  
\par 
ADA binding occurs at DPP4 cysteine-rich region,
Leu294 and Val341 on DPP4 have been demonstrated via mutagenesis to be critical for ADA-DPP4 binding.~\cite{Abbott_1999} Leu294 and Val341 are located on two short $\alpha$-helices (291-294, 341-343 respectively), those two $\alpha$-helices are suspected to play important role in ADA binding. 